\documentclass{aip-cp}

\usepackage[numbers,sort&compress]{natbib}
\usepackage{rotating}
\usepackage{graphicx}
\usepackage[utf8]{inputenc}
\usepackage[russian]{babel}
\usepackage{xcolor}

%\makeatletter
%\def\@fnsymbol#1{\ensuremath{\ifcase#1\or *\or \dagger\or **\or
%   \ddagger\or \mathsection\or \mathparagraph\or \|\or \dagger\dagger
%   \or \ddagger\ddagger \or\mathsection\mathsection
%   \or \mathparagraph\mathparagraph \or *{*}*\or
%   \dagger{\dagger}\dagger \or\ddagger{\ddagger}\ddagger\or
%   \mathsection{\mathsection}\mathsection
%   \or \mathparagraph{\mathparagraph}\mathparagraph \else\@ctrerr\fi}}
%\makeatother

% Document starts
\begin{document}

% Title portion
\title{Shock-wave Thickness Influence to the Light Diffraction on a Plane Shock Wave}

\author[aff1,aff2]{Maksim Timokhin\corref{cor1}}
\author[aff1]{Mikhail Tikhinov}
\author[aff1]{Irina Mursenkova}
\author[aff1]{Irina Znamenskaya}

\affil[aff1]{Lomonosov Moscow State University, 119991, Moscow, Russia}
\affil[aff2]{Moscow Aviation Institute, 125993, Moscow, Russia}
\corresp[cor1]{Corresponding author: timokhin@physics.msu.ru}
%\authornote[note1]{This is an example of first authornote.}
%\authornote[note2]{This is an example of second authornote.}

\maketitle


\begin{abstract}

\end{abstract}

% Head 1
\section{INTRODUCTION}

Визуализация газовых течений. Теневые методы и проч. , книжки по теневым методам (добавить из тетрадки)

Дифракция света на ударной волне и ударноволновых конфигурациях

Ссылки на Сыщикову (добавить их), Panda \cite{Panda_1995} (+ ещй две)

Книжка и статья Hornig'a (добавить) - отражение света от фронта ударной волны

Сказать про электронный пучок как неоптический вариант визуализации (Muntz, Alsmeyer, Schmidt) 

Слова про цифровую обработку и проч. 

Структура плоской ударной волны 

эксперимент \cite{Schmidt_1969, Alsmeyer_1976, Pham-Van-Diep624}, 

аналитика \cite{Becker_1922, Mott-Smith_1951, Salwen_1964}, 

численные расчёты \cite{Kogan, Dodulad_Tcheremissine_2013, Rykov2008, ShockWaves_2015, Struchtrup_Torrilhon_2004, overshoot_2015} 

\section{PROBLEM FORMULATION AND MATHEMATICAL MODEL}

Сделать картинку с учётом внутренней стреутуры волны

Набрать формул аналогичных статье 1970

Дифракция - интерференция вторичных источников
\begin{equation}
E = \sum_{j=1}^{n} A_j \left(\frac{z_0}{r_j} \right)^{0.5} e^{i \frac{2\pi}{\lambda} (r_j - z_0) + \Phi_j}
\end{equation}
где 
\begin{equation}
r_j^2= z_0^2 +( x_j - x_p)^2
\end{equation}
Здесь надо вставить рисунок поясняющий где $x_j, x_p, r_j$ 
\subsection{EXPERIMENTAL SETUP}

Описание из диссертации Орлова 2010 года

\section{RESULTS}
\subsection{Experimental case}

Сказать, что воспроизведены в точности результаты 1970 г. 

Сравнение с экспериментальной картинкой 

\subsection{The dependence on shock thickness}

Несколько картинок при одном и том же значении числа Маха (?). Лучше для двух (слабая и сильная ударные волны). Одна экспериментальная с $Ma=2.1$ и какой-нибудь $Ma=8.0$.

Итоговые графики, иллюстрирующие зависимость от длины свободного пробега --> толщины.

\section{CONCLUSION}

% Acknowledgement
\section{ACKNOWLEDGMENTS}
The work carried out at Moscow State University was supported by the Russian Science Foundation (Grant No. ).

% References

\nocite{*}
\bibliographystyle{aipnum-cp}%
\bibliography{sample}%


\end{document}